\section*{\sc Academic Projects}
\vspace{-2mm}
\hrulefill
\vspace{1mm}

\cventry
{Relevant Information Retrieval from Text Documents}
{IITK}
{\texttt{\href{https://github.com/sunil-dhaka/IR-Project}{\faGithub{} sunil-dhaka/IR-Project}}}
{Apr 2022}
{
  \begin{itemize}
    \item Implemented and fine tuned text summarization algorithms like \textbf{TextRank, LSA, Google PEGASUS, t5-small}
  %   % \item Over 230k news articles from both Hindi and English were included in the Text Corpora, along with their own human-written summaries
    \item Preprocessed corpora using varity of NLP techniques like \textbf{tokenization, stemming, lemmetization}
  %   \item Used ROUGE-N score to evaluate accuracy of the algorithm summarized text against human text summaries for documents in the dataset
  %   % \item Tested out all the algorithms on both Hindi and English language datasets and achieved SOTA f1-scores on our datasets for both abstractive algorithms after fine tuning
  %   \item Achieved \textbf{SOTA f1-scores} after fine tuning on both corpus
  %   % \item Worked in a 5 person team to do all the research, presentation and implementation work
  %   \item Created \textbf{Flask web app} for users to interact with system%the algorithms and get summaries of articles or blogs in Hindi and English languages 
  % \item Implemented text summarization algorithms like \textbf{TextRank, LSA}, seq2seq architectures like \textbf{Google PEGASUS, t5-small}
  % \item Over 230k news articles from both Hindi and English were included in the Text Corpora, along with their own human-written summaries
  % \item Preprocessed text corpora using a varity of \textbf{NLP techniques} like \textbf{tokenization, stemming, lemmetization}
  \item Used \textbf{ROUGE-N} to evaluate accuracy of machine summaries against human summaries for \textbf{corpus of size 230k}
  % \item Tested out all the algorithms on both Hindi and English language datasets and achieved SOTA f1-scores on our datasets for both abstractive algorithms after fine tuning
  \item Tested out algorithms and \textbf{achieved SOTA f1-scores} on our datasets for abstractive algorithms \textbf{after fine tuning}
  % \item Worked in a 5 person team to do all the research, presentation and implementation work
  \item Created a \textbf{Flask web application} for users to interact with the algorithms and get summaries of articles or blogs%in Hindi and English languages 
  
\end{itemize}
}

\cventry
{Racial Disparity in COVID-19 Vaccination in US}
{IME, IITK}
{\texttt{\href{https://github.com/sunil-dhaka/ime692-project}{\faGithub{} sunil-dhaka/ime692-project}}}
{Nov 2021}
{
  \begin{itemize}
    \item Analysed vaccination rates for \textbf{covid-19 by race in US counties}, along with their association with socioeconomic factors, using a number of statistical techniques
    \item Transformed and pre-processed data by \textbf{z-score} normalization and \textbf{PCA} dimension reduction of feature matrix
    \item Implemented variety of regression algorithms like \textbf{MLR, Support Vector Regression, Random Forest}
    \item \textbf{Fine tuned} the hyper parameters of these algorithms using train and test \textbf{MSE as metric} of measure  
  \end{itemize}
}

\cventry
{Indian Agriculture Data Mining}
{CSE, IITK}
{\texttt{\href{https://github.com/sunil-dhaka/Agriculture-Analysis-on-Indian-States}{\faGithub{} sunil-dhaka/Agriculture-Analysis}}}
{Nov 2021}
{
  \begin{itemize}
    \item Analysed \textbf{seasonal crop data} for last 20 years and unearthed surprising information about the Indian agricultural
    % \item Collected around 2.5 lakh data points from data-gov servers using multiple API calls
    \item Pre-processed 2.5 lakh data points using \textbf{cleaning techniques} to get robust dataset for subsequent steps
    \item Developed new \textbf{categorical variables} of zones and crop to make analysis more understandable and comparative
    \item Used \textbf{geopandas} library to discover the geographical situation of Indian agriculture on zone and crop basis
    % \item Unearthed some unexpected information regarding the Indian agricultural state
  \end{itemize}
}

\cventry
{Modelling and Forecasting of Time Series Data}
{MTH, IITK}
{\texttt{\href{https://github.com/sunil-dhaka/time-series-model}{\faGithub{} sunil-dhaka/time-series-model}}}
{Nov 2020}
{
  \begin{itemize}
    \item Modeled a \textbf{seasonal ARIMA model} to forecast next years temperatures based on historical data 
      % \item Obtained monthly average temperatures for India during the past 117 years in Celsius from data-gov website
    \item Checked data stationarity through \textbf{Dickey-Fuller test}, and performed seasonal differencing to ensure stationary TS
    \item Optimized the model parameters using \textbf{Box Jenkins Method} for forecasting purposes
    \item Performed residual analysis and \textbf{information criterion tests} to check model adequacy on the dataset
    \item Yielded absolute \textbf{MAE of 2.2\%} on test data for optimized S-ARIMA model  
  \end{itemize}
}

\vspace{-2mm}

